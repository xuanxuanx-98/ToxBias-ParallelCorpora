% This must be in the first 5 lines to tell arXiv to use pdfLaTeX, which is strongly recommended.
\pdfoutput=1
% In particular, the hyperref package requires pdfLaTeX in order to break URLs across lines.

\documentclass[11pt]{article}

% Remove the "review" option to generate the final version.
\usepackage{./figs-misc/acl2023}
\pagestyle{plain}

% Standard package includes
\usepackage{times}
\usepackage{latexsym}

% For proper rendering and hyphenation of words containing Latin characters (including in bib files)
\usepackage[T1]{fontenc}
% For Vietnamese characters
% \usepackage[T5]{fontenc}
% See https://www.latex-project.org/help/documentation/encguide.pdf for other character sets

% This assumes your files are encoded as UTF8
\usepackage[utf8]{inputenc}

% This is not strictly necessary, and may be commented out.
% However, it will improve the layout of the manuscript,
% and will typically save some space.
\usepackage{microtype}

% This is also not strictly necessary, and may be commented out.
% However, it will improve the aesthetics of text in
% the typewriter font.
\usepackage{inconsolata}

\usepackage{linguex}
\usepackage{csquotes}

% If the title and author information does not fit in the area allocated, uncomment the following
%
%\setlength\titlebox{<dim>}
%
% and set <dim> to something 5cm or larger.

\title{Dialectic Bias in Toxicity Detection of Google's Perspective API
\bigbreak A study with five parallel corpora}

% Author information can be set in various styles:
% For several authors from the same institution:
% \author{Author 1 \and ... \and Author n \\
%         Address line \\ ... \\ Address line}
% if the names do not fit well on one line use
%         Author 1 \\ {\bf Author 2} \\ ... \\ {\bf Author n} \\
% For authors from different institutions:
% \author{Author 1 \\ Address line \\  ... \\ Address line
%         \And  ... \And
%         Author n \\ Address line \\ ... \\ Address line}
% To start a seperate ``row'' of authors use \AND, as in
% \author{Author 1 \\ Address line \\  ... \\ Address line
%         \AND
%         Author 2 \\ Address line \\ ... \\ Address line \And
%         Author 3 \\ Address line \\ ... \\ Address line}

\author{Hanxin Xia $\vert$ 3417418\\
  \href{mailto://hanxin.xia@uni-duesseldorf.de}{hanxin.xia@uni-duesseldorf.de}}

\begin{document}
\maketitle

\begin{abstract}
Hate Speech detection has been a heated topic in NLP/NLU since the spread of social media. Despite the existence of various automatic toxicity scoring models, their biases against language varieties that align with minority identities have been discussed by researchers \citep{sap-etal-2019-risk}. In this work, we focus on Google's Perspective API and analyze its toxicity scores on four synthetic corpora with different dialect features compared to an original master corpus. We discovered that Perspective scored all parallel corpora significantly differently than the master corpus, which further confirms the existence of the biases. Moreover, we discovered the capping phenomenon with the toxicity scores of the gold toxic sentences: Gold non-toxic sentences are more liketly to suffer from toxicity score increase, when converted to a dialect, compared to gold toxic sentences.
\end{abstract}


\section{Introduction}

With the increasing popularity of social media and the daily widening user bases, the moderation of the contents on online communication platforms have been a crucial part of maintaining the friendliness for the users and building healthy communities. In pursuit of this objective, the platforms have implemented various means including reactive methods such as community guidelines, user report and post-removal of harmful contents and proactive methods either with sensitive words and language pattern rules \citep{gitari-2015-lexicon} or machine learning algorithms for automated hate speech detection \citep{alrehili-2019-survey}.

However current detoxifying approaches are proven to be not perfect. While the reactive methods are commonly subject to deficiency and passivity, the automatic proactive methods often discriminate against minority aligned language varieties other than Standard American English (SAE) \citep{sap-etal-2019-risk, zhou-etal-2021-challenges}.

This research aims to test the performance of Google’s Perspective API on five parallel corpora. Despite being a continuously improving model on toxicity scoring \citep{google-perspective}, the bias against non-standard English is still present in the results. The model scored (state: March 2024) texts from the four English dialects: African American Vernacular English (AAVE), Nigerian and Indian dialects and Singaporean English (Singlish). significantly differently than the original texts of standard English. Thus, we posit that there is a continued need for optimization in the hate speech detection of online contents for better mitigating biases that may infringe upon the online representation of individuals from marginalized communities.


\section{Background}

The process of detoxification and hate speech detection is aimed to create healthy and welcoming online communities. Yet the biases in these processes are leading to not only unfair penalties for minority individuals \citep{davidson-etal-2019-racial} but also potential complete exclusion of their voices in online spaces \citep{blodgett-2017-racial} simply because of the usage of non-standard language.

The underlying reasons behind such biases can be traced back to multiple factors. As \citet{xia-etal-2020-demoting} points out, first-hand annotators are inclined to classify non-abusive AAE texts as toxic, which may lead to bias already at the corpus creation stage. Secondly, surface markers on the language such as mention of specific identity terms and usage of certain uncommon patterns are also prone to cause the so-called \enquote{key word bias} in models \citep{resende-2024-comprehensive, schafer-2023-bias}. The lack of data of English language varieties in the training data may lead to their incapability to capture the real intentions and deep meanings of such speeches. Last but not the least, suboptimal performances of hate speech detection models can also be attributed to the ignorance of linguistic subtleties engrained in contextual factors such as speaker identity, environment and even special spelling \citep{davidson-etal-2019-racial}.

Several methodologies have been proposed to mitigate these biases. \citet{schafer-2023-bias} experimented with transformer based multi-class prediction in order to establish a more comprehensive viewpoint for hate speech detection with less success. Moreover, the approach of masking out identity terms exhibits obvious drawbacks despites its merits. By raising subjectivity level when identity terms occur, \citep{zhao-2022-subjectivity} were able to efficiently neutralize their model’s bias against minority aligned texts. Similar effects can be achieved by including confident dialect prediction \citep{ball-2021-differential}. By implementing adversarial learning during the model learning process, \citet{xia-etal-2020-demoting} and \citep{okpala-2022-aaebert} were both able to reduce baseline model’s false positive rate facing African American English like texts.


\section{Data}

\subsection{HateXplain}















































\newpage

\subsection{Synthetic datasets}

\subsection{Data choice}

\section{Methodology}

\section{Results}

\section{Conclusion}




\newpage




% Entries for the entire Anthology, followed by custom entries
\bibliography{references}
\bibliographystyle{./figs-misc/acl_natbib}



\appendix

\section{Code Access}

Source code and detailed instructions for conducting analyses are publicly accessible under: \href{https://github.com/xuanxuanx-98/ToxBias-ParallelCorpora}{https://github.com/xuanxuanx-98/ToxBias-ParallelCorpora}.


\end{document}
